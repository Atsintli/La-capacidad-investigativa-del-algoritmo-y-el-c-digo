%Propuesta inicial de la formación del cap # Performatividad de la(s) máquina(s) + Código + IA + Live Coding + Agencialidad, Emilio, Aaron, Marianne, Hernani, Iván, Pablo, Dorian.

%Hay algunos errores en el formato, esperamos que Emilio los corrija, please! 

\documentclass[12pt, spanish, oneside]{book} %twocolumn

\usepackage[spanish,activeacute]{babel}
\selectlanguage{spanish}
\usepackage[utf8]{inputenc}


%\usepackage{minted} %para introducir codigo
\usepackage{	}% subsubsecciones
\usepackage{hyperref}
\usepackage{float} % Allows putting an [H] in \begin{figure} to specify the exact location of the figure
\setcounter{secnumdepth}{4}
\usepackage{mdwlist}
\usepackage{natbib}
\usepackage{bibentry}
\nobibliography*
\usepackage{geometry} % Required to change the page size to A4
\geometry{
	%paper=a4paper, % Change to letterpaper for US letter
	inner=2cm, % Inner margin
	outer=2cm, % Outer margin
	bindingoffset=.5cm, % Binding offset
	top=2.5cm, % Top margin
	bottom=2.5cm, % Bottom margin
	%showframe, % Uncomment to show how the type block is set on the page
}

\usepackage{tikz} % para iteraciones
\usepackage{xstring}

\usepackage{fancyhdr}
\pagestyle{fancy}
\fancyhf{}
\fancyfoot[LE,RO]{\thepage}
%\fancyfoot[LO,RE]{User's Guide \copyright\ Copyright Information}
\fancyhead[RE]{\sffamily\nouppercase{\rightmark}}
\fancyhead[LO]{\sffamily\nouppercase{\leftmark}}
\renewcommand{\headrulewidth}{0.1pt}

%autor
\author{Aaron, Hernani}

%Interlineado
\renewcommand{\baselinestretch}{1.2}

%Para sangria
\setlength{\parindent}{2em}

%Para parrafos
\setlength{\parskip}{0pt}

%Para viudas y huerfanas (?)
\clubpenalty=10000
\widowpenalty=10000

%para imágenes
\usepackage{graphicx}

%titulo del capítulo
\title{La capacidad investigativa del algoritmo y el código: trazos sobre el ciclo práctica artística, tecnología e investigación}  
\graphicspath{ {./images/} } % path de imagenes
\begin{document}
\maketitle

\maketitle
%\tableofcontents

%\input{sections/introduccion}

% lee las secciones desde una carpeta y los incluye en orden numérico
\foreach \k in {0, ...,99} {% chktex 11 chktex 1  chktex 26
\edef\FileName{./sections/subCap\k.tex}% The % here are necessary to eliminate any
\IfFileExists{\FileName}{%  spurious spaces that may get inserted
\input{\FileName}%
    }
}

%\chapter{Conclusiones}

Blablabla...
 
\section{si necesitamos subsecciones}

bla bla bla...

%\chapter{Anexos}

Por si necesitamos anexos...
 
\section{si necesitamos subsecciones}

bla bla bla...


% lee los archivos bibs para las citas
\xdef\bibfiles{}
\foreach \k in {0, ...,99}{% chktex 11 chktex 1  chktex 26
  \edef\FileName{./bib/bib\k}%     The % here are necessary to eliminate any
    \IfFileExists{\FileName.bib}{%  spurious spaces that may get inserted
\xdef\bibfiles{\FileName,\bibfiles}%       at these points
    }
}
\StrGobbleRight{\bibfiles}{1}[\bibfiles_]

% genera la bibliografía
%\bibliographystyle{apalike}%
\bibliographystyle{plainnat}
\bibliography{\bibfiles_}
\end{document}
